PROYECTO: PRUEBAS UNITARIAS CON JEST – ADMINISTRADOR DE TAREAS
📁 CARPETAS Y ARCHIVOS DEL PROYECTO


Administrador de tareas/
├── node_modules/
├── babel.config.js
├── package.json
├── taskManager.js
├── taskManager.test.js
📁 CARPETA: node_modules/
📌 ¿Qué es?
Esta carpeta se crea automáticamente cuando usamos npm install.

🧠 ¿Para qué sirve?
Contiene todas las dependencias necesarias para ejecutar Jest y Babel.

Nunca se edita manualmente.

No se sube a GitHub ni se comparte; se genera automáticamente.

📄 ARCHIVO: package.json
📌 ¿Qué es?
Es el archivo principal de configuración del proyecto en Node.js.

🧠 ¿Para qué sirve?
Define el nombre del proyecto, versión y tipo de módulos.

Guarda las dependencias necesarias (como jest, babel-jest).

Permite ejecutar scripts como npm test.

📄 Código :
json
Copiar
Editar
{
  "name": "admin-de-tareas",              // Nombre del proyecto
  "version": "1.0.0",                      // Versión
  "type": "module",                        // Permite usar import/export
  "scripts": {
    "test": "jest"                         // Permite ejecutar pruebas con "npm test"
  },
  "devDependencies": {
    "jest": "^29.0.0",                     // Herramienta para pruebas
    "@babel/preset-env": "^7.0.0",         // Convierte código moderno
    "babel-jest": "^29.0.0"                // Conecta Jest con Babel
  }
}
📄 ARCHIVO: babel.config.js
📌 ¿Qué es?
Es el archivo que configura Babel para que pueda entender código moderno (como import/export) dentro de Jest.

🧠 ¿Para qué sirve?
Permite que Jest pueda interpretar correctamente nuestros módulos JavaScript (ESModules).

📄 Código explicado:
js

export default {
  presets: [['@babel/preset-env', { targets: { node: 'current' } }]]
};
@babel/preset-env: convierte el código moderno a una versión que Node.js pueda ejecutar.

targets: { node: 'current' }: asegura compatibilidad con tu versión de Node.

📄 ARCHIVO: taskManager.js
📌 ¿Qué es?
Este archivo contiene toda la lógica del administrador de tareas. Aquí se implementan las funciones que permiten crear, eliminar y manejar tareas.

🧠 ¿Para qué sirve?
Separar la lógica del sistema para que se pueda probar fácilmente sin depender de HTML o del navegador.

📄 Código y explicación:
js

export let tasks = [];
📌 Es un arreglo que almacena todas las tareas. Se exporta para que otros archivos puedan usarlo.

js
Copiar
Editar
export function addTask(title, date) {
  if (!title || !date) throw new Error("Campos incompletos");
📌 La función addTask crea una tarea nueva. Si falta algún campo, lanza un error para evitar datos incompletos.

js
Copiar
Editar
  const newTask = {
    id: Date.now(),
    title,
    date,
    completed: false
  };
📌 Crea el objeto newTask con un ID único, título, fecha y estado completed en false.

js
Copiar
Editar
  tasks.push(newTask);
  return newTask;
}
📌 La tarea se guarda en el arreglo tasks y se retorna para confirmar su creación.

js
Copiar
Editar
export function toggleTask(id) {
  const task = tasks.find(t => t.id === id);
  if (task) task.completed = !task.completed;
}
📌 Busca una tarea por su ID y cambia su estado: de completada a pendiente, o viceversa.

js
Copiar
Editar
export function deleteTask(id) {
  tasks = tasks.filter(t => t.id !== id);
}
📌 Elimina una tarea por su ID, dejando todas las demás.

js
Copiar
Editar
export function getTasks(filter = "all") {
  if (filter === "completed") return tasks.filter(t => t.completed);
  if (filter === "pending") return tasks.filter(t => !t.completed);
  return tasks;
}
📌 Filtra las tareas según el estado:

"all" → todas las tareas.

"completed" → solo completadas.

"pending" → solo pendientes.

js
Copiar
Editar
export function clearTasks() {
  tasks = [];
}
📌 Limpia todas las tareas. Se usa para reiniciar el sistema en las pruebas.

📄 ARCHIVO: taskManager.test.js
📌 ¿Qué es?
Este archivo contiene las pruebas unitarias usando Jest.
Cada prueba verifica que una función del sistema funcione correctamente.

🧠 ¿Para qué sirve?
Asegura que el sistema sea confiable.

Evita errores en funciones clave.

Permite hacer cambios sin miedo a romper otras partes del código.

📄 Código y explicación:
js
Copiar
Editar
import {
  addTask,
  toggleTask,
  deleteTask,
  getTasks,
  clearTasks
} from './taskManager.js';
📌 Importamos todas las funciones que vamos a probar.

js
Copiar
Editar
beforeEach(() => {
  clearTasks();
});
📌 Antes de cada prueba, limpiamos el arreglo de tareas para que los resultados no se mezclen.

✅ Prueba 1: Añadir tarea correctamente
js
Copiar
Editar
test('añade una tarea correctamente', () => {
  const task = addTask("Estudiar Jest", "2025-05-22");
  expect(task.title).toBe("Estudiar Jest");
  expect(task.date).toBe("2025-05-22");
  expect(task.completed).toBe(false);
});
✔ Verifica que la función addTask cree la tarea con los valores correctos.

✅ Prueba 2: Campos vacíos
js
Copiar
Editar
test('lanza error si falta título o fecha', () => {
  expect(() => addTask("", "")).toThrow("Campos incompletos");
});
✔ Confirma que no se puede crear una tarea sin datos válidos.

✅ Prueba 3: Cambiar estado
js
Copiar
Editar
test('cambia el estado de completado', () => {
  const task = addTask("Completar tarea", "2025-05-22");
  toggleTask(task.id);
  expect(getTasks()[0].completed).toBe(true);
});
✔ Verifica que el estado de la tarea se actualiza correctamente.

✅ Prueba 4: Eliminar tarea
js
Copiar
Editar
test('elimina una tarea por id', () => {
  const task = addTask("Tarea a eliminar", "2025-05-22");
  deleteTask(task.id);
  expect(getTasks()).toHaveLength(0);
});
✔ Asegura que la tarea desaparezca del sistema.

✅ Prueba 5: Filtrar tareas
js
Copiar
Editar
test('filtra tareas completadas y pendientes', () => {
  const t1 = addTask("Tarea 1", "2025-05-22");
  const t2 = addTask("Tarea 2", "2025-05-22");
  toggleTask(t1.id);

  expect(getTasks("completed")).toHaveLength(1);
  expect(getTasks("pending")).toHaveLength(1);
});
✔ Comprueba que el filtro devuelve correctamente las tareas según su estado.

✅ ¿CÓMO EJECUTAR LAS PRUEBAS?
Abre la terminal.

Escribe:

bash
Copiar
Editar
npm install
npm test
✅ Si todo funciona bien, verás algo así:

bash
Copiar
Editar
PASS  ./taskManager.test.js
✓ añade una tarea correctamente
✓ lanza error si falta título o fecha
✓ cambia el estado de completado
✓ elimina una tarea por id
✓ filtra tareas completadas y pendientes